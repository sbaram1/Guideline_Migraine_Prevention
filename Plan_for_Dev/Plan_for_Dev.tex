
\documentclass{paper}

\usepackage{kotex}
\usepackage{setspace}

\setstretch{1.2}


\begin{document}
\Large \textbf{개발계획 수립} \normalsize

\section{기존 진료지침 검색}
	\begin{itemize}
		\item Selection criteria
			\begin{itemize}
				\item From 2012 to 2016
				\item Written in English or Korean
				\item By Multi-disciplinary team
				\item Evidence-based method
			\end{itemize}
		\item 9개의 진료지침 검토
		\begin{itemize}
			\item 2012 AAN AHS guideline update for migraine prophylaxis Neurology (김병수, 정필욱)
			\item 2012 Canadian guideline for migraine prophylaxis Can J neurol sci (이미지, 손종희)
			\item 2012 Croatia guideline (송태진, 손종희)
			\item 2012 Danish Guidelines JHP (서종근, 최윤주)
			\item 2012 French guidelines revised JHP (정필욱, 김병수ˆ˜)
			\item 2012 Italian Guidelines revised version -JHP supple (최윤주, 서종근) 
			\item ICSI guideline 2013 (original version, full-text) (송태진, 김병수)
			\item NICE guideline for headache in over 12s (2012) (2015 update) (이미지, 김병건)
			\item Practice guideline update summary fot Botulinum neurotoxin (AAN) 2016 neurology (이미지)
		\end{itemize}
	\end{itemize}
	
\section{개발방법 결정}

\textbf{Adaptation}
\begin{itemize}
	\item 기존 국내외 진료지침이 전체의 핵심질문을 모두 포함하는 경우
	\item 관련 국내외 진료지침이 3-5년 이내에 개발되었고 결정적인 추가 근거가 없는 경우
(근거 팽창 속도가 빠른 경우 3년)
	\item 관련 국내외 진료지침(혹은 systematic review)이 근거기반 방법론*을 사용한 경우
	\item 관련 국내외 진료지침이 국가(다국가 포함) 혹은 대표적인 학회에서 개발한 경우
\end{itemize}

\end{document}