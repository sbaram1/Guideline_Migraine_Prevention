
\documentclass{paper}

\usepackage{kotex}
\usepackage{setspace}

\setstretch{1.2}

\begin{document}
\Large \textbf{개발그륩} \normalsize

\section{개발그룹 결성}
\subsection{운영위원회, Steering group}
정재면, 박광열

\subsection{자문위원회}
김병건, 김원주

\subsection{외부자문그룹}
이유경

\subsection{실무위원회}
김병수, 서종근, 손종희, 송태진, 이미지, 정필욱, 최윤주\\[5ex]
추가 고려: \\
다학제성 고려 (FM, IM etc), \\
방법론 전문가(문헌검색, Systematic review)

\subsection{근거평가그룹}

\subsection{행정조직}

\vspace{30pt}
\section{이해관계 선언}
%USPSTF 이해상충에 대한 기준: 금액등

\subsection{이해 상충의 종류}
\begin{itemize}
	\item 재정적 이해  상충: 연구비, 자문료, 사례비, 주식, 지적재산권등
	\item 지적 이해 상충: 진료 헤게모니 등 
	\item 개인적인 이해 상충
%	\item 업무 활동에 있어 이해 상충(Conflicts of commitment)
\end{itemize}

\subsection{이해 상충에 대한 처리}
\begin{itemize}
	\item 공개
	\item 위원회에서 배제
	\item 해당 권고의 결정권 배제
\end{itemize}

\vspace{30pt}
\section{운영}
\subsection{합의 원칙의 결정}
토론으로 결정\\
델파이방법?

\subsection{저자 원칙 결정}
초안: \\
최종보고서: 정재면\\
저자 포함: 개발그룹 전체\\
저자 순서: 

\subsection{잠재적 승인기구 선정}
대한 두통학회\\
대한 신경과학회 (외부 검토시 포함)

\subsection{보급및 실행전략}
두통학회/신경과학회 홈페이지 게시\\
두통학회지, 신경과학회지 발표

\end{document}